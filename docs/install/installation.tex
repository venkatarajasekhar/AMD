\section{Installation}

\textbf{AMD} is configured, compiled and installed using the Python-based
\textbf{Waf}%
\footnote{Waf - The meta build system - Google Project Hosting. Retrieved
January 27, 2015, from \texttt{https://code.google.com/p/waf/}} platform.

\subsection{Why Waf?}

The following quote on the Black Duck Open Hub review of Waf%
\footnote{The Waf Open Project on Open Hub. Retrieved January 27, 2015, from
\texttt{https://www.openhub.net/p/waf}} best describes the reason why we
decided to choose Waf:
\begin{itemize}
\item Waf depends on Python only which is ported on most platforms 
\item Waf scripts are Python modules which are easier to learn and to maintain
that custom languages 
\item Waf license has very little constraints (BSD) and can be redistributed
easily (all in one 100kb script) 
\item Waf architecture is modular and can be extended easily, it relies
on state-of-the-art algorithms 
\item Waf provides many more features than its competitors 
\item Waf provides many small projects and code snippets
\end{itemize}

\subsection{System requirements}
\begin{itemize}
\item \textbf{git} 
\item \textbf{Python} 2.6 or 2.7
\item \textbf{Python C libraries }(e.g., \texttt{libpython2.7})
\item C and \Cpp{} compilers supporting \Cpp{}03 standard.
\item \textbf{GNU make: make }in Linux,\textbf{ }or\textbf{ gnumake }in
darwin-MacOSX 
\item \textbf{Boost}%
\footnote{Boost \Cpp{} Libraries. Retrieved January 29, 2015, from
http://www.boost.org/} 
\end{itemize}

\subsection{Configure}

After download, \texttt{cd} to the top-level directory, then run one
of the following commands:

\begin{itemize}
\item To build the libraries and the examples, run: \texttt{\$ ./waf configure}

\item To build the libraries only, run: \texttt{\$ ./waf configure
-{}-bld-AMD}

\item To build the examples, run:

\begin{description}
\item [{\texttt{\$}}] \texttt{./waf configure -{}-bld-examples}
\end{description}
\item To build the tests, run:

\begin{description}
\item [{\texttt{\$}}] \texttt{./waf configure -{}-bld-AMD -{}-tests}
\end{description}

\item To build Python bindings, run

\begin{description}
\item [{\texttt{\$}}] \texttt{./waf configure -{}-bld-python}
\end{description}

\item Individual examples, packages, and dependencies can also be configured
to be build, and command line flags can be combined. Run the following
command for more information about available options:

\begin{description}
\item [{\texttt{\$}}] \texttt{./waf -{}-help}
\end{description}
\end{itemize}

\subsection{Build}

To actually execute the build after running one of the configuration
commands above, run 
\begin{description}
\item [{\texttt{\$}}] \texttt{./waf build}
\end{description}
See Appendix \ref{sec:Troubleshooting} for possible solutions in
case the build command above does not succeed.

\subsection{Installing packages}

Most of the required third party dependencies can be automatically
installed as follows. After downloading NLP<Go>, \texttt{cd} to the
top-level directory and run the command:
\begin{description}
\item [{\texttt{\$}}] \texttt{./waf configure -{}-install-deps }
\end{description}
