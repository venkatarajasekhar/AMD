\section{Theoretical Contributions}
\label{sec:theory}

\begin{frame}
\frametitle{Theoretical Contributions}
\framesubtitle{Quick Recap}
%
So far, we have seen:
\begin{itemize}
\item Automatic differentiation of scalars/vectors.
\item Forward- and reverse-mode differentiation.
\item Definition of derivatives for:
  \begin{itemize}
  \item scalar-matrix and matrix-matrix functions.
  \end{itemize}
\item Direct matrix products
  \begin{itemize}
  \item Kronecker and matrix-outer products (known).
  \item \alert{Box products (new contribution)}.
  \end{itemize}
\end{itemize}
%
\end{frame}

\begin{frame}
\frametitle{Theoretical Contributions}
\framesubtitle{Designing AMD}
%
\begin{itemize}
\item Design box product.
  \begin{itemize}
  \item Needed to compute all 24 direct matrix products.
  \end{itemize}
\item Establish derivatives for the functions we care about:
  \begin{itemize}
\item Scalar-matrix: {\color{m1}trace} and {\color{m1}log-determinant}.
\item Matrix-matrix:
  $\color{m1}+$, $\color{m1}-$, $\color{m1}*$, $\color{m1}.*$, 
                              {\color{m1}transpose} and {\color{m1}inverse}.
  \end{itemize}
\item Establish chain rule for function compositions.
\item Figure out the mode:
  \begin{itemize}
  \item Forward-mode (intuitive but expensive).
  \item matrix-matrix (counter-intuitive but efficient).
  \end{itemize}
\item Do everything without blowing up memory.
\end{itemize}
%
\end{frame}

\subsection{The Box Product}

\begin{frame}
\frametitle{Box Product}
\framesubtitle{Definition}
Let $\color{m1}\mA\in\R^{m_1\times n_1}$ and
$\color{m1}\mB\in\R^{m_2\times n_2}$

\begin{definition}[Kronecker Product]
$\color{m1}\mA\otimes\mB\in\R^{(m_1m_2)\times(n_1n_2)}$
is defined by
{\color{m1}$(\mA\otimes\mB)_{(i-1)m_2+j,(k-1)n_2+l}=$ $a_{i\alert{k}}b_{j\alert{l}}=$ $(\mA\otimes\mB)_{(ij)(kl)}$}.
\end{definition}

\begin{definition}[Box Product]
$\color{m1}\mA\boxtimes\mB\in\R^{(m_1m_2)\times(n_1n_2)}$
is defined by
{\color{m1}
$(\mA\boxtimes\mB)_{(i-1)m_2+j,(k-1)n_1+l}=$ $a_{i\alert{l}}b_{j\alert{k}}=$ $(\mA\boxtimes\mB)_{(ij)(kl)}$.}
\end{definition}
\end{frame}

\begin{frame}
\frametitle{Box Product}
\framesubtitle{An example Kronecker and box product}
Consider two $\color{m1}2\times 2$ matrices, $\color{m1}\mA$ and
$\color{m1}\mB$:
\footnotesize
$$
\color{m1}
\begin{array}{cc}
\mA\otimes\mB & \mA\boxtimes\mB\\
\begin{pmatrix}
{\color<5>{OliveGreen}{a_{11}}}\color<1>{OliveGreen}{b_{11}} & 
{\color<6>{OliveGreen}{a_{11}}}\color<1>{OliveGreen}{b_{12}} & 
{\color<5>{OliveGreen}{a_{12}}}\color<2>{OliveGreen}{b_{11}} & 
{\color<6>{OliveGreen}{a_{12}}}\color<2>{OliveGreen}{b_{12}} \\
%
{\color<7>{OliveGreen}{a_{11}}}\color<1>{OliveGreen}{b_{21}} &
{\color<8>{OliveGreen}{a_{11}}}\color<1>{OliveGreen}{b_{22}} &
{\color<7>{OliveGreen}{a_{12}}}\color<2>{OliveGreen}{b_{21}} & 
{\color<8>{OliveGreen}{a_{12}}}\color<2>{OliveGreen}{b_{22}} \\
%
{\color<5>{OliveGreen}{a_{21}}}\color<3>{OliveGreen}{b_{11}} & 
{\color<6>{OliveGreen}{a_{21}}}\color<3>{OliveGreen}{b_{12}} & 
{\color<5>{OliveGreen}{a_{22}}}\color<4>{OliveGreen}{b_{11}} & 
{\color<6>{OliveGreen}{a_{22}}}\color<4>{OliveGreen}{b_{12}} \\
%
{\color<7>{OliveGreen}{a_{21}}}\color<3>{OliveGreen}{b_{21}} & 
{\color<8>{OliveGreen}{a_{21}}}\color<3>{OliveGreen}{b_{22}} & 
{\color<7>{OliveGreen}{a_{22}}}\color<4>{OliveGreen}{b_{21}} & 
{\color<8>{OliveGreen}{a_{22}}}\color<4>{OliveGreen}{b_{22}} \\
\end{pmatrix} 
& 
\begin{pmatrix}
{\color<5>{BrickRed}{a_{11}}}\color<1>{BrickRed}{b_{11}} & 
{\color<5>{BrickRed}{a_{12}}}\color<2>{BrickRed}{b_{11}} & 
{\color<6>{BrickRed}{a_{11}}}\color<1>{BrickRed}{b_{12}} & 
{\color<6>{BrickRed}{a_{12}}}\color<2>{BrickRed}{b_{12}} \\
%
{\color<7>{BrickRed}{a_{11}}}\color<1>{BrickRed}{b_{21}} & 
{\color<7>{BrickRed}{a_{12}}}\color<2>{BrickRed}{b_{21}} & 
{\color<8>{BrickRed}{a_{11}}}\color<1>{BrickRed}{b_{22}} & 
{\color<8>{BrickRed}{a_{12}}}\color<2>{BrickRed}{b_{22}} \\
%
{\color<5>{BrickRed}{a_{21}}}\color<3>{BrickRed}{b_{11}} & 
{\color<5>{BrickRed}{a_{22}}}\color<4>{BrickRed}{b_{11}} & 
{\color<6>{BrickRed}{a_{21}}}\color<3>{BrickRed}{b_{12}} & 
{\color<6>{BrickRed}{a_{22}}}\color<4>{BrickRed}{b_{12}} \\
%
{\color<7>{BrickRed}{a_{21}}}\color<3>{BrickRed}{b_{21}} & 
{\color<7>{BrickRed}{a_{22}}}\color<4>{BrickRed}{b_{21}} & 
{\color<8>{BrickRed}{a_{21}}}\color<3>{BrickRed}{b_{22}} & 
{\color<8>{BrickRed}{a_{22}}}\color<4>{BrickRed}{b_{22}} \\
%
\end{pmatrix} 
\end{array}
$$
\end{frame}

\begin{frame}
\frametitle{Box Product}
\framesubtitle{Some Identities}
%
The box product behaves similarly to the Kronecker product:
\begin{enumerate}
\item \textbf{Vector Multiplication}:
{\footnotesize
$$\color{m1}
\begin{array}{cc}
(\mB^\top\otimes\mA)\myvec(\mX)=\myvec(\mA\mX\mB) &
(\mB^\top\boxtimes\mA)\myvec(\mX)=\myvec(\mA\mX^{\alert{\top}}\mB)
\end{array}
$$}
\item \textbf{Matrix Multiplication}:
{\footnotesize
$$\color{m1}
\begin{array}{cc}
(\mA\otimes\mB)(\mC\otimes\mD)=(\mA\mC)\otimes(\mB\mD)&
(\mA\boxtimes\mB)(\mC\boxtimes\mD)=(\mA\alert{\mD})\alert{\otimes}(\mB\alert{\mC})
\end{array}
$$}
\item \textbf{Inverse and Transpose}:
{\footnotesize
$$\color{m1}
\begin{array}{cc}
(\mA\otimes\mB)^{-1}=\mA^{-1}\otimes\mB^{-1} &
(\mA\boxtimes\mB)^{-1}=\alert{\mB}^{-1}\boxtimes\alert{\mA}^{-1}\\
(\mA\otimes\mB)^{\top}=\mA^{\top}\otimes\mB^{\top} &
(\mA\boxtimes\mB)^{\top}=\alert{\mB}^{\top}\boxtimes\alert{\mA}^{\top}
\end{array}
$$}
\item \textbf{Mixed Products}:
{\footnotesize
$$\color{m1}
\begin{array}{cc}
(\mA\otimes\mB)(\mC\boxtimes\mD) =
  (\mA\alert{\mC})\boxtimes(\mB\alert{\mD}) &
(\mA\boxtimes\mB)(\mC\otimes\mD) =
  (\mA\alert{\mD})\boxtimes(\mB\alert{\mC})
\end{array}
$$}
\end{enumerate}
\end{frame}

\begin{frame}
\frametitle{Box Product}
\framesubtitle{Some Identities}
Let $\color{m1}\mA\in\R^{m_1\times n_1}$, $\color{m1}\mB\in\R^{m_2\times n_2}$ then
\begin{itemize}
\item \textbf{Trace:}
{\footnotesize
$$\color{m1}
\trace(\mA\otimes\mB) =
\trace(\mA)\trace(\mB)\qquad\trace(\mA\boxtimes\mB) = \trace(\mA\mB)
$$
}
\item \textbf{Determinant:}  Here $m_1=n_1$ and $m_2=n_2$ is required
{\footnotesize
\color{m1}
\begin{eqnarray*}
\det(\mA\otimes\mB) &=&(\det(\mA))^{m_2}(\det(\mA))^{m_1}\\
\det(\mA\boxtimes\mB) &=&
\alert{(-1)^{{m_1 \choose 2}{m_2\choose 2}}}(\det(\mA))^{m_2}(\det(\mB))^{m_1}
\end{eqnarray*}}
\item \textbf{Associativity:}
{\footnotesize
$$\color{m1}
(\mA\otimes\mB)\otimes\mC=
\mA\otimes(\mB\otimes\mC) \qquad 
(\mA\boxtimes\mB)\boxtimes\mC=
\mA\boxtimes(\mB\boxtimes\mC),
$$
}
but not for mixed products.  In general we have
{\footnotesize
$$\color{m1}
(\mA\otimes\mB)\boxtimes\mC \alert{\neq}
\mA\otimes(\mB\boxtimes\mC) \qquad 
(\mA\boxtimes\mB)\otimes\mC \alert{\neq}
\mA\boxtimes(\mB\otimes\mC).
$$
}
%
\end{itemize}

\end{frame}

\begin{frame}
\frametitle{Box Product}
\framesubtitle{Identity Box Products}
$\color{m1}\mI_m\boxtimes\mI_n$ is permutation matrix; 
%
Let $\color{m1}\mA\in\R^{m_1\times n_1}$, $\color{m1}\mB\in\R^{m_2\times n_2}$:
%
\begin{description}
\item[Orthonormal:]
$\color{m1}(\mI_m\boxtimes\mI_n)^\top(\mI_m\boxtimes\mI_n)=\mI_{mn}$
\item[Transposition:]
$\color{m1}(\mI_{m_1}\boxtimes\mI_{n_1})\myvec(\mA) = \myvec(\mA^\top)$
\item[Connector:]
Converting a Kronecker product to a box product:
$\color{m1}(\mA\otimes\mB)(\mI_{n_1}\boxtimes\mI_{n_2}) =
\mA\boxtimes\mB$.\\
Converting a box product to a Kronecker product:
$\color{m1}(\mA\boxtimes\mB)(\mI_{n_2}\boxtimes\mI_{n_1}) =
\mA\otimes\mB$.
\end{description}
\end{frame}

\begin{frame}
\frametitle{Box Product}
\framesubtitle{A Wolf in Sheep's Clothing}
\begin{itemize}
\item Notation for identity box products is new, but:
  \begin{itemize}
  \item Physics: 
    $\color{m1}\mI_m\boxtimes\mI_n \rightarrow{} \color{m1}\mT_{m,n}$
  \item Computer Science: 
    $\color{m1}\mI_m\boxtimes\mI_n \rightarrow{}$ \alert{stride permutation}
  \item Statistics:
    $\color{m1}\mI_m\boxtimes\mI_n \rightarrow{}$ \alert{perfect shuffle}
  \end{itemize}
\item Box-product can express complex identities compactly.
\end{itemize}
%
\end{frame}


\subsection{Setting Up Identities}
\begin{frame}
\frametitle{Scalar-Matrix Derivatives}
First some simple $\trace$ identities:
$$\color{m1}
\begin{array}{ll}
f(\mX)=\trace(\mX) & \qquad f'(\mX) = \mI \\
f(\mX)=\trace(\mA\mX) & \qquad f'(\mX) = \mA^\top \\
f(\mX)=\trace(\mA\mX^\top) & \qquad f'(\mX) = \mA 
\end{array}
$$
Now, some simple $\logdet$ identities:
$$\color{m1}
\begin{array}{ll}
f(\mX)=\logdet(\mX) & \qquad f'(\mX) = \mX^{-\!\top}  \\
f(\mX)=\logdet(\mA\mX) & \qquad f'(\mX) = \mX^{-\!\top} 
\end{array}
$$
\end{frame}

\begin{frame}
\frametitle{Matrix-Matrix Derivative}
\framesubtitle{Basic Differentiation Identities}
Let $\color{m1}\mX\in\R^{m\times n}$.
\begin{itemize}
\item \textbf{Identity}: 
{\footnotesize
  $$\color{m1}\mF(\mX)=\mX, \mF'(\mX)=\mI_{mn}$$
}
\item \textbf{Transpose}:
{\footnotesize
  $$\color{m1}\mF(\mX)=\mX^\top, \mF'(\mX)=\mI_{m}\boxtimes\mI_n$$
}
\item \textbf{Chain Rule}: 
{\footnotesize 
$$\color{m1}\mF(\mX)=\mG(\mH(\mX)),\quad\mF'(\mX)=\mG'(\mH(\mX))\mH'(\mX)$$
}
\item \textbf{Product Rule}: 
{\footnotesize
$$\color{m1}\mF(\mX) = \mG(\mX)\mH(\mX), \quad\mF'(\mX) =
(\mI\otimes\mH^\top(\mX)) \mG'(\mX)+(\mG(\mX)\otimes\mI)\mH'(\mX)$$
}
\end{itemize}
\end{frame}

\begin{frame}
\frametitle{Derivative Identitities}
Assume $\color{m1}\mX\in\R^{m\times m}$ is a square matrix.
\begin{itemize}
\item \textbf{Square:}
$\color{m1}\mF(\mX)=\mX^2$, $\color{m1}\mF'(\mX)=\mI_m\otimes\mX^\top+\mX\otimes\mI_m$.
\item \textbf{Inverse:} $\color{m1}\mF(\mX)=\mX^{-1}$, $\color{m1}\mF'(\mX)=-\mX^{-1}\otimes\mX^{-\!\top}$.
\item \textbf{+Transpose:} $\color{m1}\mF(\mX)=\mX^{-\!\top}$, $\color{m1}\mF'(\mX)=-\mX^{-\!\top}\boxtimes\mX^{-1}$.
\item \textbf{Square Root:} $\color{m1}\mF(\mX)=\mX^{1/2}$,
{\footnotesize
  $\color{m1}\mF'(\mX) = 
  \left(\mI\otimes(\mX^{1/2})^\top+ \mX^{1/2}\otimes\mI \right)^{-1}$.
}
%\item \textbf{Exponential:}
% $\color{m1}\mF(\mX)=\exp(\mX)$,
%  $\color{m1}\mF'(\mX)=(e^\mX\otimes\mI-\mI\otimes
%  e^{\mX^\top})\left(\mX\otimes\mI-\mI\otimes\mX^\top\right)^{-1}$. \alert{new}
\item \textbf{Integer Power:} $\color{m1}\mF(\mX)=\mX^k$,
  $\color{m1}\mF'(\mX)=\sum_{i=0}^{k-1}\mX^i\otimes(\mX^{k-1-i})^{\top}$.
\end{itemize}
\end{frame}

\subsection{Setting Up Chain-Rule}
\begin{frame}
\frametitle{Matrix-Matrix Derivative}
\framesubtitle{Setting Up Chain Rule($\mF(\mX) = \trace(\mX)$)}
%
\footnotesize
\textcolor{m1}{
\begin{align*}
\mF(\mX) &= \trace(\mX) \\
\mF'(\mX) &= \frac{\partial}{\partial\mX}(\trace(\mX)) 
              \frac{\partial\mX}{\partial\mX} \\
          &= \mI \left(\mI\otimes{}\mI\right)\\
          &= \R^{m\times{}m}(\R^{m^2\times{}m^2})\\
\end{align*}
} % textcolor
\normalsize
\vspace{-30pt}
\begin{itemize}
\item This does not make any sense.
\item We know that: 
  \begin{itemize}
  \item For scalar-matrix functions: 
                              $\color{m1}\mF'(\mX) \in{} \R^{m\times{}n}$
  \item For matrix-matrix functions: 
                              $\color{m1}\mF'(\mX) \in{} \R^{m^2\times{}n^2}$
  \end{itemize}
\end{itemize}
\end{frame}

\begin{frame}
\frametitle{Matrix-Matrix Derivative}
\framesubtitle{Setting Up Chain Rule($\mF(\mX) = \trace(\mX)$)}
%
\footnotesize
\colorbox{green!10}{\vbox{
\textcolor{m1}{
\begin{align*}
\myvec\left(\mF'(\mG(\mX))^\top \right) &= 
               \left(\mG'(\mX)\right)^\top
\myvec\left(
      \left(\frac{\partial}{\partial\mG(\mX)}(\mF(\mG(\mX)))\right)^\top
      \right)
\end{align*}
} % textcolor
} }
\normalsize
\footnotesize
\textcolor{m1}{
\begin{align*}
\mF(\mX) &= \trace(\mX) \\
\myvec \left(\left(\mF'(\mX)\right)^\top\right) &= 
       \left(\left(\frac{\partial\mX}{\partial\mX}\right)^\top\right)
    \myvec\left(
      \left(\frac{\partial}{\partial\mX}(\trace(\mX))\right)^\top
    \right) \\
          &= \left(\mI\otimes{}\mI\right)^\top \myvec\left(\mI^\top\right)\\
          &= \left(\mI\otimes{}\mI\right) \myvec\left(\mI\right) \\
          &= \myvec\left(\mI\right) \\
\mF'(\mX) &= \mI \in{} \R^{m\times{}m}
\end{align*}
} % textcolor
\normalsize
%
\end{frame}

\begin{frame}
\frametitle{Matrix-Matrix Derivative}
\framesubtitle{Setting Up Chain Rule ($\mF(\mX) = \logdet(\mX)$)}
%
\footnotesize
\colorbox{green!10}{\vbox{
\textcolor{m1}{
\begin{align*}
\myvec\left(\mF'(\mG(\mX))^\top \right) &= 
               \left(\mG'(\mX)\right)^\top
\myvec\left(
      \left(\frac{\partial}{\partial\mG(\mX)}(\mF(\mG(\mX)))\right)^\top
      \right)
\end{align*}
} % textcolor
} }
\normalsize
\footnotesize
\textcolor{m1}{
\begin{align*}
\mF(\mX) &= \logdet(\mX) \\
\myvec \left(\left(\mF'(\mX)\right)^\top\right) &= 
       \left(\left(\frac{\partial\mX}{\partial\mX}\right)^\top\right)
    \myvec\left(
      \left(\frac{\partial}{\partial\mX}(\logdet(\mX))\right)^\top
    \right) \\
          &= \left(\mI\otimes{}\mI\right)^\top \myvec\left(
                              \left(\mX^{-\!\top}\right)^\top\right)\\
          &= \left(\mI\otimes{}\mI\right) \myvec\left( \mX^{-1} \right)
          = \myvec\left( \mX^{-1} \right)\\
 \mF'(\mX)&= \mX^{-\!\top} \in{} \R^{m\times{}m}
\end{align*}
} % textcolor
\normalsize
%
\end{frame}

%\begin{frame}
%\frametitle{Human Scalar-Matrix Differentiation}
%The chain rule to get scalar-matrix derivatives is awkward to use.
%Instead we have some short-cuts.\\
%\medskip
%$$\color{m1}
%\frac{\partial }{\partial
%  \mX}\trace(\mG(\mX)\mH(\mX))=
%\left.\frac{\partial}{\partial
%  \mX}
%\trace\left(\mH(\mY)\mG(\mX)+\mH(\mX)\mG(\mY)\right)\right|_{\mY=\mX},
%$$
%$$\color{m1}
%\frac{\partial}{\partial \mX}
%\trace(\mA\mF^{-1}(\mX))
%= \left.-\frac{\partial}{\partial
%  \mX}
%\trace\left(\mF^{-1}(\mY)\mA\mF^{-1}(\mY)\mF(\mX)\right)\right|_{\mY=\mX},
%$$
%and
%$$\color{m1}
%\frac{\partial \logdet(\mF(\mX))}{\partial
%  \mX} =
%\left.\frac{\partial}{\partial
%  \mX} \trace\left(\mF^{-1}(\mY)\mF(\mX)\right)\right|_{\mY=\mX}.
%$$
%\end{frame}

%\begin{frame}
%Finally if 
%$$\color{m1}
%r(x)=\frac{q(x)}{p(x)}=\frac{\sum_{i=1}^na_i x^i}{\sum_{j=1}^nb_ix^i}
%$$
%is a scalar-scalar function we can form a matrix-matrix function by simply
%substituting a matrix $\color{m1}\mX$ for the scalar $\color{m1}x$.
%Then 
%$$\color{m1}
%\frac{\partial}{\partial\mX}\trace(r(\mX)) = \left(r'(\mX)\right)^\top
%$$
%and if $\color{m1}h(x)=\log(r(x))$ then
%$$\color{m1}
%\frac{\partial}{\partial\mX}\logdet(r(\mX)) = \left(h'(\mX)\right)^\top.
%$$
%\end{frame}

\subsection{Forward and Reverse Mode Differentiation}
\begin{frame}
\centerline{Compute $\color{m1}f(\mX)=\trace(\mF(\mX))=\trace(((\mI+\mX)^{-1}\mX^\top)\mX)$}
\begin{center}
\begin{tikzpicture}
[scale=.7,transform shape,
matrix/.style={rectangle,draw=black,fill=blue!20,thick},
operation/.style={circle,draw=black,fill=red!20,thick},
arrow/.style={->,thick}]
\node[matrix] (F) {$\trace(\mF(\mX))$};
\node[left=5cm of F] (upperLeft) { };
\node[right=2cm of F] (upperRight) { };
\node [right,left=2pt of F]  (tmpF) at (F.west)
{
\only<7->{$\color{blue}f=\trace(\mT_5)$}
};\node[operation] (root) [below=of F] {$\times$}
edge [arrow] (F);
\node [right,left=2pt of root] (tmpRoot)
{
\only<6->{$\color{blue}\mT_5=\mT_4*\mX$}
};
\node[operation] (t4) [below left=of root] {$\times$}
edge [arrow] (root);
\node [left,align=center,red] (tmp4) at (t4.west)
{
\only<5->{$\color{blue}\mT_4=\mT_2*\mT_3$}
};
\node[matrix] (x3) [below right=of root] {$\mX$}
edge [arrow] (root);
\node[operation] (t2) [below left=of t4] {$(\cdot)^{-1}$}
edge [arrow] (t4);
\node [left,align=center,red] (tmp2) at (t2.west)
{
\only<3->{$\color{blue}\mT_2=\mT_1^{-1}$}
};
\node[operation] (t3) [below right=of t4] {$(\cdot)^\top$}
edge [arrow] (t4);
\node [left=2pt of t3]
{
\only<4->{$\color{blue}\mT_3=\mX^\top$}
};
\node[operation] (t1) [below=of t2] {$+$}
edge [arrow] (t2);
\node[matrix] (x2) [below=of t3] {$\mX$}
edge [arrow] (t3);
\node[matrix] (x1) [below right=of t1] {$\mX$}
edge [arrow] (t1);
\node[matrix] (i) [below left=of t1] {$\mI$}
edge [arrow] (t1);
\node [left,align=center,red] (textt1) at (t1.west)
{ \only<2->{$\color{blue}\mT_1=\mI+\mX$} };
%\begin{pgfonlayer}{background}
%\node [inner sep=.1cm, fill=green!10, fit=(i) (x3) (F) (upperLeft) (upperRight)] {};
%\end{pgfonlayer}
\end{tikzpicture}
\end{center}
\end{frame}

\begin{frame}
\vbox to .3cm {
\only<1>{\centerline{Forward mode computation of
    $\color{m1}f'(\mX)$:}}
\only<2>{\centerline{$\color{m1}(\mG+\mH)'=\mG'+\mH'$}}
\only<3>{\centerline{Chain rule:
    $\color{m1}(\mT_1^{-1})'=-(\mT_1^{-1}\otimes\mT_1^{-\!\top})\mT_1'$ }}
\only<4>{\centerline{$\color{m1}(\mX^\top)'=\mI\boxtimes\mI$}}
\only<5-6>{\centerline{Product rule:
    $\color{m1}(\mG\mH)'=(\mI\otimes\mH^\top)\mG'+(\mG\otimes\mI)\mH'$}} 
\only<7>{\centerline{$\color{m1}((\trace(\mF))')^\top=\myvec^{-1}\left(\myvec^\top(\mI)\mF'\right)$}}
}
\begin{center}
\begin{tikzpicture}
[scale=.68,transform shape,
matrix/.style={rectangle,draw=black,fill=blue!20,thick},
operation/.style={circle,draw=black,fill=red!20,thick},
arrow/.style={->,thick}]
\node[matrix] (F) {$\trace(\mF(\mX))$};
\node[left=7cm of F] (upperLeft) { };
\node[right=2cm of F] (upperRight) { };
\node[operation] (root) [below=of F] {$\times$}
edge [arrow] (F);
\node [left=2pt of F] (f6) {
\only<7->{\color{Sepia}\alert<7>{
$f'=\myvec^{-1}\left(\myvec^\top(\mI)\mT_5'\right)$
}}};
\node [right=2pt of root] (tmpRoot)
{ $\color{blue}\mT_5$ };
\node [left=2pt of root] (f5) {
\only<6->{\color{Sepia}\alert<6>{
$\mT_5'=(\mI\otimes\mX^\top)\mT_4'+(\mT_4\otimes\mI)(\mI\otimes\mI)$
}}};
\node[operation] (t4) [below left=of root] {$\times$}
edge [arrow] (root);
\node [right=2pt of t4] (tmp4)
{ $\color{blue}\mT_4$ };
\node [left=2pt of t4] (f4) {
\only<5->{\color{Sepia}\alert<5>{
$\mT_4'=(\mI\otimes\mT_3^\top)\mT_2'+(\mT_2\otimes\mI)\mT_3'$
}}};
\node[matrix] (x3) [below right=of root] {$\mX$}
edge [arrow] (root);
\node [left=2pt of x3] { \color{Sepia}\alert<1-5>{$\mI\otimes\mI$} };
\node[operation] (t2) [below left=of t4] {$(\cdot)^{-1}$}
edge [arrow] (t4);
\node [right=2pt of t2]
{ $\color{blue}\mT_2$ };
\node [left=2pt of t2] (f2) { 
  \only<3->{\color{Sepia}\alert<3-4>{$\mT_2'=-(\mT_2\otimes\mT_2^\top)\mT_1'$}} 
};
\node[operation] (t3) [below right=of t4] {$(\cdot)^\top$}
edge [arrow] (t4);
\node [left=2pt of t3]
{ $\color{blue}\mT_3$ };
\node [right=2pt of t3] (f3) {
  \only<4->{\color{Sepia}\alert<4>{ $\mT_3'=\mI\boxtimes\mI$ }}
};
\node[operation] (t1) [below=of t2] {$+$}
edge [arrow] (t2);
\node[matrix] (x2) [below=of t3] {$\mX$}
edge [arrow] (t3);
\node [left=2pt of x2] { \color{Sepia}\alert<1-3>{$\mI\otimes\mI$} };
\node [right=2pt of t1]
{ $\color{blue}\mT_1$ };
\node [left=2pt of t1] (f1) {
  \only<2->{\color{Sepia}\alert<2>{$\mT_1'=\m0+\mI\otimes\mI$}} 
};
\node[matrix] (x1) [below right=of t1] {$\mX$}
edge [arrow] (t1);
\node [left=2pt of x1] { \color{Sepia}\alert<1>{$\mI\otimes\mI$} };
\node[matrix] (i) [below left=of t1] {$\mI$}
edge [arrow] (t1);
\node [left=2pt of i] (n0) { \color{Sepia}\alert<1>{$\m0$} };
%\begin{pgfonlayer}{background}
%\node [inner sep=.1cm, fill=green!10, fit=(i) (x3) (F) (n0) (upperLeft) (upperRight)] {};
%\end{pgfonlayer}
\end{tikzpicture}
\end{center}
\end{frame}


\begin{frame}
Forward mode differentiation
\begin{itemize}
\item Requires huge intermediate matrices.
\item Only the last step reduces the size.
\end{itemize}
\textbf{Key points:} 
\begin{itemize}
\item $\color{m1}\mF'(\mX)$ is composed of Kronecker, box, and outer products.
  \begin{itemize}
  \item Only for a large class of functions.
  \end{itemize}
\item These can be ``unwound'':
$$\color{m1}
\myvec^\top(\mC)\mA\otimes\mB = \myvec^\top(\mB^\top\mC\mA)
$$
\end{itemize}
%
\begin{center}
\vspace{-20pt}
{\color{red}Formation of huge intermediate matrices can be eliminated.}
\end{center}
%
%Reverse mode differentiation:
%\begin{itemize}
%\item Evaluate the derivative from top to bottom.
%\item Small scalar-matrix derivatives are propagated down the tree.
%\end{itemize}
\end{frame}

\begin{frame}
\frametitle{Reverse Mode Differentiation}
\framesubtitle{Transpose}
\footnotesize
\colorbox{green!10}{\vbox{
\textcolor{m1}{
\begin{align*}
\myvec\left(\mF'(\mG(\mX))^\top \right) &= 
               \left(\mG'(\mX)\right)^\top
\myvec\left(
      \left(\frac{\partial}{\partial\mG(\mX)}(\mF(\mG(\mX)))\right)^\top
      \right)
\end{align*}
} % textcolor
} }
\normalsize
\begin{minipage}{0.2\textwidth}
\begin{center}
\begin{tikzpicture}
[scale=.7,transform shape,
matrix/.style={rectangle,draw=black,fill=blue!20,thick},
operation/.style={circle,draw=black,fill=red!20,thick},
arrow/.style={<-,thick}]
\node[matrix] (x1) {$\mG(\mX)$};
\node[operation, above=2cm of x1] (f1) {$(\cdot)^\top$} 
edge [arrow] (x1);
\node[left=0.1cm of f1] (textf1) {$\color{m1}\mZ$};
\node[left=0.1cm of x1] (textx1) {$\color{m2}\mZ^\top$};
\end{tikzpicture}
\end{center}
\end{minipage}
\begin{minipage}{0.5\textwidth}
\footnotesize
{\color{m1}
\begin{align*}
        \mF'\left(\mG(\mX)\right) &= 
          \left(\mI\boxtimes\mI\right)\mG'(\mX); \mF(\mY) = \mY^\top \\
\frac{\partial}{\partial\mX}\left((\mG(\mX))^\top\right)^\top
           \myvec\left(\mZ^\top\right) &= 
     \left(\left(\mI\boxtimes\mI\right)\mG'(\mX)\right)^\top
           \myvec\left(\mZ^\top\right) \\
           &=
          \left(\mG'(\mX)\right)^\top
          \left(\mI\boxtimes\mI\right)^\top\myvec\left(\mZ^\top\right) \\
           &=
          \left(\mG'(\mX)\right)^\top
          \left(\mI\boxtimes\mI\right)\myvec\left(\mZ^\top\right) \\
           &=
     {\color{m2}
          \left(\mG'(\mX)\right)^\top
          \myvec\left(\mZ\right)} \\
\end{align*}
}
\normalsize
\end{minipage}
\end{frame}

\begin{frame}
\frametitle{Reverse Mode Differentiation}
\framesubtitle{Inverse}
\footnotesize
\colorbox{green!10}{\vbox{
\textcolor{m1}{
\begin{align*}
\myvec\left(\mF'(\mG(\mX))^\top \right) &= 
               \left(\mG'(\mX)\right)^\top
\myvec\left(
      \left(\frac{\partial}{\partial\mG(\mX)}(\mF(\mG(\mX)))\right)^\top
      \right)
\end{align*}
} % textcolor
} }
\normalsize
\begin{center}
\begin{tikzpicture}
[scale=.7,transform shape,
matrix/.style={rectangle,draw=black,fill=blue!20,thick},
operation/.style={circle,draw=black,fill=red!20,thick},
arrow/.style={<-,thick}]
\node[matrix] (x1) {$\mG(\mX)$};
\node[operation, above=2cm of x1] (f1) {$(\cdot)^{-1}$} 
edge [arrow] (x1);
\node[left=0.1cm of f1] (textf1) {$\color{m1}\mZ$};
\node[left=0.1cm of x1] (textx1) 
      {$\color{m2}\mF(\mG(\mX))^\top\mZ\mF(\mG(\mX))^\top$};
\end{tikzpicture}
\end{center}
\end{frame}

\begin{frame}
\frametitle{Reverse Mode Differentiation}
\framesubtitle{Sum}
\footnotesize
\colorbox{green!10}{\vbox{
\textcolor{m1}{
\begin{align*}
\myvec\left(\mF'(\mG(\mX))^\top \right) &= 
               \left(\mG'(\mX)\right)^\top
\myvec\left(
      \left(\frac{\partial}{\partial\mG(\mX)}(\mF(\mG(\mX)))\right)^\top
      \right)
\end{align*}
} % textcolor
} }
\normalsize
\begin{center}
\begin{tikzpicture}
[scale=.7,transform shape,
matrix/.style={rectangle,draw=black,fill=blue!20,thick},
operation/.style={circle,draw=black,fill=red!20,thick},
arrow/.style={<-,thick}]
\node[matrix] (x1) {$\mG(\mX)$};
\node[matrix, left=2cm of x1] (x2) {$\mH(\mX)$};
\node[right=1cm of x2] (rx2) {};
\node[operation, above=2cm of rx2] (f1) {$+$} 
edge [arrow] (x1) edge [arrow] (x2);
%
\node[left=0.1cm of f1] (textf1) {$\color{m1}\mZ$};
\node[right=0.1cm of x1] (textx1) {$\color{m2}\mZ$};
\node[left=0.1cm of x2] (textx2) {$\color{m2}\mZ$};
\end{tikzpicture}
\end{center}
%
\end{frame}

\begin{frame}
\frametitle{Reverse Mode Differentiation}
\framesubtitle{Minus}
\footnotesize
\colorbox{green!10}{\vbox{
\textcolor{m1}{
\begin{align*}
\myvec\left(\mF'(\mG(\mX))^\top \right) &= 
               \left(\mG'(\mX)\right)^\top
\myvec\left(
      \left(\frac{\partial}{\partial\mG(\mX)}(\mF(\mG(\mX)))\right)^\top
      \right)
\end{align*}
} % textcolor
} }
\normalsize
\begin{center}
\begin{tikzpicture}
[scale=.7,transform shape,
matrix/.style={rectangle,draw=black,fill=blue!20,thick},
operation/.style={circle,draw=black,fill=red!20,thick},
arrow/.style={<-,thick}]
\node[matrix] (x1) {$\mG(\mX)$};
\node[matrix, left=2cm of x1] (x2) {$\mH(\mX)$};
\node[right=1cm of x2] (rx2) {};
\node[operation, above=2cm of rx2] (f1) {$-$} 
edge [arrow] (x1) edge [arrow] (x2);
%
\node[left=0.1cm of f1] (textf1) {$\color{m1}\mZ$};
\node[right=0.1cm of x1] (textx1) {$\color{m2}-\mZ$};
\node[left=0.1cm of x2] (textx2) {$\color{m2}\mZ$};
\end{tikzpicture}
\end{center}
\end{frame}

\begin{frame}
\frametitle{Reverse Mode Differentiation}
\framesubtitle{Product}
\footnotesize
\colorbox{green!10}{\vbox{
\textcolor{m1}{
\begin{align*}
\myvec\left(\mF'(\mG(\mX))^\top \right) &= 
               \left(\mG'(\mX)\right)^\top
\myvec\left(
      \left(\frac{\partial}{\partial\mG(\mX)}(\mF(\mG(\mX)))\right)^\top
      \right)
\end{align*}
} % textcolor
} }
\normalsize
\begin{center}
\begin{tikzpicture}
[scale=.7,transform shape,
matrix/.style={rectangle,draw=black,fill=blue!20,thick},
operation/.style={circle,draw=black,fill=red!20,thick},
arrow/.style={<-,thick}]
\node[matrix] (x1) {$\mG(\mX)$};
\node[matrix, left=2cm of x1] (x2) {$\mH(\mX)$};
\node[right=1cm of x2] (rx2) {};
\node[operation, above=2cm of rx2] (f1) {$\times{}$} 
edge [arrow] (x1) edge [arrow] (x2);
%
\node[left=0.1cm of f1] (textf1) {$\color{m1}\mZ$};
\node[right=0.1cm of x1] (textx1) {$\color{m2}\mH(\mX)^\top\mZ$};
\node[left=0.1cm of x2] (textx2) {$\color{m2}\mZ\mG(\mX)^\top$};
\end{tikzpicture}
\end{center}
\end{frame}

\begin{frame}
\frametitle{Reverse Mode Differentiation}
\framesubtitle{Element-wise Product}
\footnotesize
\colorbox{green!10}{\vbox{
\textcolor{m1}{
\begin{align*}
\myvec\left(\mF'(\mG(\mX))^\top \right) &= 
               \left(\mG'(\mX)\right)^\top
\myvec\left(
      \left(\frac{\partial}{\partial\mG(\mX)}(\mF(\mG(\mX)))\right)^\top
      \right)
\end{align*}
} % textcolor
} }
\normalsize
\begin{center}
\begin{tikzpicture}
[scale=.7,transform shape,
matrix/.style={rectangle,draw=black,fill=blue!20,thick},
operation/.style={circle,draw=black,fill=red!20,thick},
arrow/.style={<-,thick}]
\node[matrix] (x1) {$\mG(\mX)$};
\node[matrix, left=2cm of x1] (x2) {$\mH(\mX)$};
\node[right=1cm of x2] (rx2) {};
\node[operation, above=2cm of rx2] (f1) {$\circ$} 
edge [arrow] (x1) edge [arrow] (x2);
%
\node[left=0.1cm of f1] (textf1) {$\color{m1}\mZ$};
\node[right=0.1cm of x1] (textx1) {$\color{m2}\mH(\mX)\circ\mZ$};
\node[left=0.1cm of x2] (textx2) {$\color{m2}\mG(\mX)\circ\mZ$};
\end{tikzpicture}
\end{center}
\end{frame}

\begin{frame}
\vbox to .3cm {
\only<1>{
\centerline{Reverse Mode Differentiation:
$\color{m1}(f')^\top=\myvec^{-1}\left(\myvec^\top(\mI)\mF'\right)$}}
\only<2>{
\centerline{$\color{m1}\myvec^\top(\mR_0)(\mI\otimes\mX^\top)=\myvec^\top(\alert<2>{\mX\mR_0\mI})$ }
}
\only<3>{
\centerline{$\color{m1}\myvec^\top(\mR_0)(\mT_4\otimes\mI)=\myvec^\top(\alert{\mI\mR_0\mT_4})$ }
}
\only<4>{
\centerline{$\color{m1}\myvec^\top(\mR_1)(\mI\otimes\mT_3^\top)=\myvec^\top(\alert{\mT_3\mR_1\mI})$ }
}
\only<5>{
\centerline{$\color{m1}\myvec^\top(\mR_1)(\mT_2\otimes\mI)=\myvec^\top(\alert{\mI\mR_1\mT_2})$ }
}
\only<6>{
\centerline{$\color{m1}\myvec^\top(\mR_3)(-\mT_2\otimes\mT_2^\top)=\myvec^\top(\alert{-\mT_2\mR_3\mT_2})$ }
}
\only<7>{
\centerline{$\color{m1}\myvec^\top(\mR_4)(\mI\boxtimes\mI)=\myvec^\top(\alert{\mI\mR_4^\top\mI})$ }
}
\only<8>{
\centerline{$\color{m1}\myvec^\top(\mR_5)\m0=\myvec(\m0)$ }
}
\only<9>{
\centerline{$\color{m1}\myvec^\top(\mR_5)\mI\otimes\mI=\myvec(\mR_5)$ }
}
\only<10>{
\centerline{$\color{m1}f'(\mX)=\mR_2^\top+\mR_6^\top+\mR_8^\top$ }
}
}
\begin{center}
\begin{tikzpicture}
[scale=.68,transform shape,
matrix/.style={rectangle,draw=black,fill=blue!20,thick},
operation/.style={circle,draw=black,fill=red!20,thick},
arrow/.style={->,thick}]
\node[matrix] (F) {$\trace(\mF(\mX))$};
\node[left=8cm of F] (upperLeft) { };
\node[right=4cm of F] (upperRight) { };
\node[operation] (root) [below=of F] {$\times$}
edge [arrow] (F);
\node [left=2cm of F] (f6) {\color{OliveGreen}$f'=$
\only<3->{\alert<3>{$\mR_2^\top$}}
\only<7->{\alert<7>{$+\mR_6^\top$}}
\only<9->{\alert<9>{$+\mR_8^\top$}}
};
\node [right=2pt of F] (r0) {\alert<1-3>{$\mR_0=\mI$}};
\node [right=2pt of root] (tmpRoot)
{ $\color{blue}\mT_5$ };
\node [left=2pt of root] (f5) {
  \only<2-3>{\color{Sepia}$\mT_5'=$
    \alert<2>{$(\mI\otimes\mX^\top)$}
    \color{Sepia}$\mT_4'+$
    \alert<3>{$\mT_4\otimes\mI$}
  }
};
\node[operation] (t4) [below left=of root] {$\times$}
edge [arrow] (root);
\node [right=2pt of t4] (tmp4)
{ $\color{blue}\mT_4$ };
\node [above left=2pt of t4.north east] (r1) {
  \only<2-5> { \color{OliveGreen}\alert<2,4-5>{$\mR_1=\mX\mR_0$} } 
};
\node [left=2pt of t4] (f4) {
\only<4-5>{
\color{Sepia}$\mT_4'=$
\alert<4>{$(\mI\otimes\mT_3^\top)$}
\color{Sepia}$\mT_2'+$
\alert<5>{$(\mT_2\otimes\mI)$}
\color{Sepia}$\mT_3'$
}};
\node[matrix] (x3) [below right=of root] {$\mX$}
edge [arrow] (root);
\node [right=2pt of x3] {
  \only<3->{\color{OliveGreen}
    \alert<3>{$\mR_2=\mR_0\mT_4$}
  }
 };
\node[operation] (t2) [below left=of t4] {$(\cdot)^{-1}$}
edge [arrow] (t4);
\node [right=2pt of t2]
{ $\color{blue}\mT_2$ };
\node [above left=5pt of t2.north east] (r3) {
  \only<4-6> { \color{OliveGreen}\alert<4,6>{$\mR_3=\mT_3\mR_1$} } 
};
\node [left=2pt of t2] (f2) { 
  \only<6>{$\mT_2'=\alert{-(\mT_2\otimes\mT_2^\top)}\mT_1'$}
};
\node[operation] (t3) [below right=of t4] {$(\cdot)^\top$}
edge [arrow] (t4);
\node [left=2pt of t3]
{ $\color{blue}\mT_3$ };
\node [above right=5pt of t3.north west] (r3) {
  \only<5-7> { \color{OliveGreen}\alert<5,7>{$\mR_4=\mR_1\mT_2$} } 
};
\node [right=2pt of t3] (f3) {
  \only<7>{ $\mT_3'=\alert{\mI\boxtimes\mI}$ }
};
\node[operation] (t1) [below=of t2] {$+$}
edge [arrow] (t2);
\node[matrix] (x2) [below=of t3] {$\mX$}
edge [arrow] (t3);
\node [right=2pt of x2] (r3) {
  \only<7-> { \color{OliveGreen}\alert<7>{$\mR_6=\mR_4^\top$} } 
};
\node [right=2pt of t1] { $\color{blue}\mT_1$ };
\node [above left=5pt of t1.north east] (r5) {
  \only<6-9> { \color{OliveGreen}\alert<6,8-9>{$\mR_5=-\mT_2\mR_3\mT_2$} } 
};
\node [left=2pt of t1] (f1) {
  \only<8-9>{$\mT_1'=$\alert<8>{$\m0$}+\alert<9>{$\mI\otimes\mI$}}
};
\node[matrix] (x1) [below right=of t1] {$\mX$}
edge [arrow] (t1);
\node [right=2pt of x1] (r8) {
  \only<9-> { \color{OliveGreen}\alert<9>{$\mR_8=\mR_5$} } 
};
\node[matrix] (i) [below left=of t1] {$\mI$}
edge [arrow] (t1);
\node [right=2pt of i] (r7) {
  \only<8-9> { \color{OliveGreen}\alert<8>{$\mR_7=\m0$} } 
};
%\begin{pgfonlayer}{background}
%\node [inner sep=.1cm, fill=green!10, fit=(i) (x3) (F) (n0) (upperLeft) (upperRight)] {};
%\end{pgfonlayer}
\end{tikzpicture}
\end{center}
\end{frame}

\begin{frame}
\frametitle{Automatic Scalar/Vector Differentiation}
\framesubtitle{Computational Cost of Differentiation}
\begin{theorem}
The algorithm described for reverse-mode matrix differentiation can be computed
with no more than 4 times the arithmetic operations than is needed to compute
the function when the matrix operations make no assumptions about the structure
of the matrix operands (such as matrices being banded or upper triangular).
\end{theorem}
%
\begin{itemize}
\item We are working hard to ensure optimizations in our software.
\item Non-trivial to carry out all optimizations.
\end{itemize}
\end{frame}
