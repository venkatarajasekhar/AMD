\subsection{The Box Product}

\begin{frame}
\frametitle{Box Product}
\framesubtitle{Definition}
Let $\color{m1}\mA\in\R^{m_1\times n_1}$ and
$\color{m1}\mB\in\R^{m_2\times n_2}$

\begin{definition}[Kronecker Product]
$\color{m1}\mA\otimes\mB\in\R^{(m_1m_2)\times(n_1n_2)}$
is defined by
{\color{m1}$(\mA\otimes\mB)_{(i-1)m_2+j,(k-1)n_2+l}=$ $a_{i\alert{k}}b_{j\alert{l}}=$ $(\mA\otimes\mB)_{(ij)(kl)}$}.
\end{definition}

\begin{definition}[Box Product]
$\color{m1}\mA\boxtimes\mB\in\R^{(m_1m_2)\times(n_1n_2)}$
is defined by
{\color{m1}
$(\mA\boxtimes\mB)_{(i-1)m_2+j,(k-1)n_1+l}=$ $a_{i\alert{l}}b_{j\alert{k}}=$ $(\mA\boxtimes\mB)_{(ij)(kl)}$.}
\end{definition}
\end{frame}

\begin{frame}
\frametitle{Box Product}
\framesubtitle{An example Kronecker and box product}
Consider two $\color{m1}2\times 2$ matrices, $\color{m1}\mA$ and
$\color{m1}\mB$:
\footnotesize
$$
\color{m1}
\begin{array}{cc}
\mA\otimes\mB & \mA\boxtimes\mB\\
\begin{pmatrix}
{\color<5>{OliveGreen}{a_{11}}}\color<1>{OliveGreen}{b_{11}} & 
{\color<6>{OliveGreen}{a_{11}}}\color<1>{OliveGreen}{b_{12}} & 
{\color<5>{OliveGreen}{a_{12}}}\color<2>{OliveGreen}{b_{11}} & 
{\color<6>{OliveGreen}{a_{12}}}\color<2>{OliveGreen}{b_{12}} \\
%
{\color<7>{OliveGreen}{a_{11}}}\color<1>{OliveGreen}{b_{21}} &
{\color<8>{OliveGreen}{a_{11}}}\color<1>{OliveGreen}{b_{22}} &
{\color<7>{OliveGreen}{a_{12}}}\color<2>{OliveGreen}{b_{21}} & 
{\color<8>{OliveGreen}{a_{12}}}\color<2>{OliveGreen}{b_{22}} \\
%
{\color<5>{OliveGreen}{a_{21}}}\color<3>{OliveGreen}{b_{11}} & 
{\color<6>{OliveGreen}{a_{21}}}\color<3>{OliveGreen}{b_{12}} & 
{\color<5>{OliveGreen}{a_{22}}}\color<4>{OliveGreen}{b_{11}} & 
{\color<6>{OliveGreen}{a_{22}}}\color<4>{OliveGreen}{b_{12}} \\
%
{\color<7>{OliveGreen}{a_{21}}}\color<3>{OliveGreen}{b_{21}} & 
{\color<8>{OliveGreen}{a_{21}}}\color<3>{OliveGreen}{b_{22}} & 
{\color<7>{OliveGreen}{a_{22}}}\color<4>{OliveGreen}{b_{21}} & 
{\color<8>{OliveGreen}{a_{22}}}\color<4>{OliveGreen}{b_{22}} \\
\end{pmatrix} 
& 
\begin{pmatrix}
{\color<5>{BrickRed}{a_{11}}}\color<1>{BrickRed}{b_{11}} & 
{\color<5>{BrickRed}{a_{12}}}\color<2>{BrickRed}{b_{11}} & 
{\color<6>{BrickRed}{a_{11}}}\color<1>{BrickRed}{b_{12}} & 
{\color<6>{BrickRed}{a_{12}}}\color<2>{BrickRed}{b_{12}} \\
%
{\color<7>{BrickRed}{a_{11}}}\color<1>{BrickRed}{b_{21}} & 
{\color<7>{BrickRed}{a_{12}}}\color<2>{BrickRed}{b_{21}} & 
{\color<8>{BrickRed}{a_{11}}}\color<1>{BrickRed}{b_{22}} & 
{\color<8>{BrickRed}{a_{12}}}\color<2>{BrickRed}{b_{22}} \\
%
{\color<5>{BrickRed}{a_{21}}}\color<3>{BrickRed}{b_{11}} & 
{\color<5>{BrickRed}{a_{22}}}\color<4>{BrickRed}{b_{11}} & 
{\color<6>{BrickRed}{a_{21}}}\color<3>{BrickRed}{b_{12}} & 
{\color<6>{BrickRed}{a_{22}}}\color<4>{BrickRed}{b_{12}} \\
%
{\color<7>{BrickRed}{a_{21}}}\color<3>{BrickRed}{b_{21}} & 
{\color<7>{BrickRed}{a_{22}}}\color<4>{BrickRed}{b_{21}} & 
{\color<8>{BrickRed}{a_{21}}}\color<3>{BrickRed}{b_{22}} & 
{\color<8>{BrickRed}{a_{22}}}\color<4>{BrickRed}{b_{22}} \\
%
\end{pmatrix} 
\end{array}
$$
\end{frame}

\begin{frame}
\frametitle{Box Product}
\framesubtitle{Some Identities}
%
The box product behaves similarly to the Kronecker product:
\begin{enumerate}
\item \textbf{Vector Multiplication}:
{\footnotesize
$$\color{m1}
\begin{array}{cc}
(\mB^\top\otimes\mA)\myvec(\mX)=\myvec(\mA\mX\mB) &
(\mB^\top\boxtimes\mA)\myvec(\mX)=\myvec(\mA\mX^{\alert{\top}}\mB)
\end{array}
$$}
\item \textbf{Matrix Multiplication}:
{\footnotesize
$$\color{m1}
\begin{array}{cc}
(\mA\otimes\mB)(\mC\otimes\mD)=(\mA\mC)\otimes(\mB\mD)&
(\mA\boxtimes\mB)(\mC\boxtimes\mD)=(\mA\alert{\mD})\alert{\otimes}(\mB\alert{\mC})
\end{array}
$$}
\item \textbf{Inverse and Transpose}:
{\footnotesize
$$\color{m1}
\begin{array}{cc}
(\mA\otimes\mB)^{-1}=\mA^{-1}\otimes\mB^{-1} &
(\mA\boxtimes\mB)^{-1}=\alert{\mB}^{-1}\boxtimes\alert{\mA}^{-1}\\
(\mA\otimes\mB)^{\top}=\mA^{\top}\otimes\mB^{\top} &
(\mA\boxtimes\mB)^{\top}=\alert{\mB}^{\top}\boxtimes\alert{\mA}^{\top}
\end{array}
$$}
\item \textbf{Mixed Products}:
{\footnotesize
$$\color{m1}
\begin{array}{cc}
(\mA\otimes\mB)(\mC\boxtimes\mD) =
  (\mA\alert{\mC})\boxtimes(\mB\alert{\mD}) &
(\mA\boxtimes\mB)(\mC\otimes\mD) =
  (\mA\alert{\mD})\boxtimes(\mB\alert{\mC})
\end{array}
$$}
\end{enumerate}
\end{frame}

\begin{frame}
\frametitle{Box Product}
\framesubtitle{Some Identities}
Let $\color{m1}\mA\in\R^{m_1\times n_1}$, $\color{m1}\mB\in\R^{m_2\times n_2}$ then
\begin{itemize}
\item \textbf{Trace:}
{\footnotesize
$$\color{m1}
\trace(\mA\otimes\mB) =
\trace(\mA)\trace(\mB)\qquad\trace(\mA\boxtimes\mB) = \trace(\mA\mB)
$$
}
\item \textbf{Determinant:}  Here $m_1=n_1$ and $m_2=n_2$ is required
{\footnotesize
\color{m1}
\begin{eqnarray*}
\det(\mA\otimes\mB) &=&(\det(\mA))^{m_2}(\det(\mA))^{m_1}\\
\det(\mA\boxtimes\mB) &=&
\alert{(-1)^{{m_1 \choose 2}{m_2\choose 2}}}(\det(\mA))^{m_2}(\det(\mB))^{m_1}
\end{eqnarray*}}
\item \textbf{Associativity:}
{\footnotesize
$$\color{m1}
(\mA\otimes\mB)\otimes\mC=
\mA\otimes(\mB\otimes\mC) \qquad 
(\mA\boxtimes\mB)\boxtimes\mC=
\mA\boxtimes(\mB\boxtimes\mC),
$$
}
but not for mixed products.  In general we have
{\footnotesize
$$\color{m1}
(\mA\otimes\mB)\boxtimes\mC \alert{\neq}
\mA\otimes(\mB\boxtimes\mC) \qquad 
(\mA\boxtimes\mB)\otimes\mC \alert{\neq}
\mA\boxtimes(\mB\otimes\mC).
$$
}
%
\end{itemize}

\end{frame}

\begin{frame}
\frametitle{Box Product}
\framesubtitle{Identity Box Products}
$\color{m1}\mI_m\boxtimes\mI_n$ is permutation matrix; 
%
Let $\color{m1}\mA\in\R^{m_1\times n_1}$, $\color{m1}\mB\in\R^{m_2\times n_2}$:
%
\begin{description}
\item[Orthonormal:]
$\color{m1}(\mI_m\boxtimes\mI_n)^\top(\mI_m\boxtimes\mI_n)=\mI_{mn}$
\item[Transposition:]
$\color{m1}(\mI_{m_1}\boxtimes\mI_{n_1})\myvec(\mA) = \myvec(\mA^\top)$
\item[Connector:]
Converting a Kronecker product to a box product:
$\color{m1}(\mA\otimes\mB)(\mI_{n_1}\boxtimes\mI_{n_2}) =
\mA\boxtimes\mB$.\\
Converting a box product to a Kronecker product:
$\color{m1}(\mA\boxtimes\mB)(\mI_{n_2}\boxtimes\mI_{n_1}) =
\mA\otimes\mB$.
\end{description}
\end{frame}

\begin{frame}
\frametitle{Box Product}
\framesubtitle{A Wolf in Sheep's Clothing}
\begin{itemize}
\item Notation for identity box products is new, but:
  \begin{itemize}
  \item Physics: 
    $\color{m1}\mI_m\boxtimes\mI_n \rightarrow{} \color{m1}\mT_{m,n}$
  \item Computer Science: 
    $\color{m1}\mI_m\boxtimes\mI_n \rightarrow{}$ \alert{stride permutation}
  \item Statistics:
    $\color{m1}\mI_m\boxtimes\mI_n \rightarrow{}$ \alert{perfect shuffle}
  \end{itemize}
\item Box-product can express complex identities compactly.
\end{itemize}
%
\end{frame}
